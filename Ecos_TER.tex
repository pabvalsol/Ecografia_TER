\documentclass[]{article}
\usepackage{lmodern}
\usepackage{amssymb,amsmath}
\usepackage{ifxetex,ifluatex}
\usepackage{fixltx2e} % provides \textsubscript
\ifnum 0\ifxetex 1\fi\ifluatex 1\fi=0 % if pdftex
  \usepackage[T1]{fontenc}
  \usepackage[utf8]{inputenc}
\else % if luatex or xelatex
  \ifxetex
    \usepackage{mathspec}
  \else
    \usepackage{fontspec}
  \fi
  \defaultfontfeatures{Ligatures=TeX,Scale=MatchLowercase}
\fi
% use upquote if available, for straight quotes in verbatim environments
\IfFileExists{upquote.sty}{\usepackage{upquote}}{}
% use microtype if available
\IfFileExists{microtype.sty}{%
\usepackage{microtype}
\UseMicrotypeSet[protrusion]{basicmath} % disable protrusion for tt fonts
}{}
\usepackage[margin=1in]{geometry}
\usepackage{hyperref}
\hypersetup{unicode=true,
            pdftitle={Ecografía en la Agencia Sanitaria Costa del Sol. ¿Qué pasa con los técnicos?},
            pdfauthor={Pablo Valdés Solís. DAIG Radiodiagnóstico},
            pdfborder={0 0 0},
            breaklinks=true}
\urlstyle{same}  % don't use monospace font for urls
\usepackage{natbib}
\bibliographystyle{apalike}
\usepackage{longtable,booktabs}
\usepackage{graphicx,grffile}
\makeatletter
\def\maxwidth{\ifdim\Gin@nat@width>\linewidth\linewidth\else\Gin@nat@width\fi}
\def\maxheight{\ifdim\Gin@nat@height>\textheight\textheight\else\Gin@nat@height\fi}
\makeatother
% Scale images if necessary, so that they will not overflow the page
% margins by default, and it is still possible to overwrite the defaults
% using explicit options in \includegraphics[width, height, ...]{}
\setkeys{Gin}{width=\maxwidth,height=\maxheight,keepaspectratio}
\IfFileExists{parskip.sty}{%
\usepackage{parskip}
}{% else
\setlength{\parindent}{0pt}
\setlength{\parskip}{6pt plus 2pt minus 1pt}
}
\setlength{\emergencystretch}{3em}  % prevent overfull lines
\providecommand{\tightlist}{%
  \setlength{\itemsep}{0pt}\setlength{\parskip}{0pt}}
\setcounter{secnumdepth}{5}
% Redefines (sub)paragraphs to behave more like sections
\ifx\paragraph\undefined\else
\let\oldparagraph\paragraph
\renewcommand{\paragraph}[1]{\oldparagraph{#1}\mbox{}}
\fi
\ifx\subparagraph\undefined\else
\let\oldsubparagraph\subparagraph
\renewcommand{\subparagraph}[1]{\oldsubparagraph{#1}\mbox{}}
\fi

%%% Use protect on footnotes to avoid problems with footnotes in titles
\let\rmarkdownfootnote\footnote%
\def\footnote{\protect\rmarkdownfootnote}

%%% Change title format to be more compact
\usepackage{titling}

% Create subtitle command for use in maketitle
\providecommand{\subtitle}[1]{
  \posttitle{
    \begin{center}\large#1\end{center}
    }
}

\setlength{\droptitle}{-2em}

  \title{Ecografía en la Agencia Sanitaria Costa del Sol. ¿Qué pasa con los técnicos?}
    \pretitle{\vspace{\droptitle}\centering\huge}
  \posttitle{\par}
    \author{Pablo Valdés Solís. DAIG Radiodiagnóstico}
    \preauthor{\centering\large\emph}
  \postauthor{\par}
      \predate{\centering\large\emph}
  \postdate{\par}
    \date{2019-06-23}

\documentclass[11pt]{book}
\usepackage[spanish]{babel}
%\selectlanguage{spanish}
\usepackage[utf8]{inputenc}

 %para cambiar el tipo de fuente por defecto
\renewcommand{\sfdefault}{phv}
% helvética en los títulos
\renewcommand{\rmfamily}{phv}

% helvetica en el cuerpo del documento
\sffamily

\begin{document}
\maketitle

{
\setcounter{tocdepth}{2}
\tableofcontents
}
\hypertarget{introduccion}{%
\section{Introducción}\label{introduccion}}

La ecografía es una de las técnicas de imagen más solicitadas. Sus indicaciones están creciendo y su utilidad es cada vez mayor.

Simultáneamente, los precios de los equipos y su tamaño están disminuyendo, de forma que ahora hay equipos muy baratos, con calidad de imagen suficiente y que se pueden llevar a la habitación del paciente. Esto origina:

\begin{itemize}
\tightlist
\item
  Los fabricantes buscan ampliar su cartera de clientes, y para ello ofrecen equipos a servicios que previamente no hacían ecografía (como medicina interna, UCI, dermatología\ldots{})
\item
  Muchos servicios ven la ecografía bien como una necesidad para mejorar su calidad de atención, o bien como una oportunidad de negocio.
\end{itemize}

En este contexto, hay más demanda de ecografías y más oferta de ecografía por diferentes servicios. Sin embargo:

\begin{itemize}
\tightlist
\item
  No hay regulación en la formación ni en la adquisición de competencias. Muchos servicios que hacen ecografía no tienen recogida esta técnica como parte de su formación durante el periodo de residencia.
\item
  No hay regulación en cómo se debe hacer la ecografía, ni de temas tan básicos como exigir que la ecografía se acompañe de imágenes en el PACS o un informe asociado.
\end{itemize}

La situación actual, y que probablemente irá creciendo, es que la ecografía está pasando a ser una técnica que para muchos clínicos es ``como el fonendoscopio'' y que creen que se puede hacer como algo rutinario dentro de la exploración física del paciente.

\hypertarget{las-competencias-en-ecografia}{%
\section{Las competencias en ecografía}\label{las-competencias-en-ecografia}}

Como cualquier técnica de radiología, la ecografía tiene una fase de adquisición de imágenes y otra de informe del estudio. Habitualmente, los técnicos obtienen las imágenes y los radiólogos las informan. Sin embargo, en la ecografía se asume que la adquisición y el informe son dos procesos paralelos casi simultáneos y que ambas deberian ser hechas por el mismo profesional (radiólogo).

Sin embargo, no todas las ecografías son iguales. Algunas (como la ecografía cervical, la ecografía escrotal, la ecografía de caderas del neonato, los Doppler venoso de miembros inferiores o carotídeo\ldots{}) pueden estandarizarse con una serie de procedimientos que permitan que \textbf{siempre se adquieran las mismas imágenes, con los mismos criterios}, algo que se hace en la TC o la RM. En estos casos, se pueden definir los criterios para adquirir la competencia. En algunos casos, como la ecografía de abdomen, es difícil obtener unas imágenes estandarizadas y, por lo tanto, conseguir que la adquisición de imágenes y el informe de las imágenes se haga en dos fases diferentes.

Por estos motivos, es importante definir las \textbf{competencias en la adquisición de las ecografías}. La idea es definir una serie de pruebas que se puedan estandarizar y determinar los criterios de adquisición para su realización. Como ejemplo, la \emph{ecografía cervical}. Es una técnica fácil de estandarizar, en la que se organiza a definir qué imágenes hay que incluir, con sus criterios de calidad. Esto hace que la ecografía cervical sea un ejemplo de posible formación específica y de adquisición de competencias.

\hypertarget{la-formacion-en-ecografia-de-los-tecnicos}{%
\section{La formación en ecografía de los Técnicos}\label{la-formacion-en-ecografia-de-los-tecnicos}}

Se supone que el plan formativo de los técnicos incluye la ecografía, algo que sabemos que no es real. Por eso, se ha definido un proceso de formación en ecografía dirigido a obtener la competencia de ``Adquisición de imágenes ecográficas en el cuello''.

Este programa consiste en:

\begin{itemize}
\tightlist
\item
  \textbf{Formación genérica en ecografía}: curso teórico avanzado de física e imagen ecográfica. Se trata de un curso no presencial, vía internet. Los técnicos tienen que aprobarlo para pasar a la siguiente fase.
\item
  \textbf{Formación específica en ecografía cervical}: metodología similar al anterior, pero en este caso es un curso específico de ecografía cervical.
\item
  \textbf{Prácticas con simulación} para conocer el proceso y evaluar las competencias genéricas. En esta fase se explica el manejo del ecógrafo, y del proceso general, y se comprueba si el usuario es competente en temas ¨" generales como la comunicación, manejo de la sala, etc. Estas competencias evaluadas se registran, y cuando el técnIco las supera, pasa a la siguiente fase.
\item
  \textbf{Prácticas con pacientes}: inicialmente, el técnico acompaña al radiologo; observa cómo hace las ecografías y repasa los conceptos. Posteriormente, el técnico repite la ecografía, después de que la realice el radiólogo y en presencia de éste. Cuando el radiólogo revisa varios casos y comprueba que el técnico comprende el procedimiento, se cambia la dinámica y en la siguiente fase, el radiólogo hace la ecografía, el técnico la mira y posteriormente, el técnico la repite solo. En la siguiente fase, el técnico hace la ecografía y registra los cortes predeterminados. El radiólogo repite la ecografía y compara sus cortes con los que hizo el técnico. Por último, cuando el técnico ya domina el procedimiento, hace la ecografía en primer lugar y el radiólogo solo supervisa las dudas y comprueba el estudio. \textbf{En todas las fases se hace un registro de las imágenes y de los procedimientos}.
\end{itemize}

De esta forma, la competencia se consigue de forma tutelada y, al final del proceso, el técnico es capaz de adquirir imágenes de ecografía cervical.

En esta primera parte del proyecto están participando ocho técnicos, que están en distintas fases del mismo. En septiembre de 2016, al menos dos hacen la adquisición de imágenes con buena calidad.

\hypertarget{la-situacion-en-el-hospital}{%
\section{La situación en el hospital}\label{la-situacion-en-el-hospital}}

La situación en nuestro centro en lo que se refiere a la ecografía es compleja:

\begin{itemize}
\tightlist
\item
  La demanda está creciendo de forma importante, especialmente en lo que se refiere a solicitudes de atención primaria.
\item
  Muchos servicios han adquirido ecógrafos con fondos propios de I + D. Están haciendo ecografías, en muchos casos de dudosa calidad, sin recoger las imágenes en el PACS y sin hacer informes correspondientes (en el mejor de los casos, se incluye una referencia en la historia del paciente).
\item
  La formación de estos médicos se está haciendo a partir de cursos impartidos por sus propios especialistas. De hecho, los últimos cursos de formación en ecografía impartidos en nuestro hospital fueron organizados por los servicios de Medicina Interna y de Dermatología.
\end{itemize}

En lo que se refiere a las listas de espera, la situación es insostenible, con demoras importantes y un número de pacientes pendientes de cita que sobrepasa los límites razonables. Con la situación actual de salas y personal no es previsible que se pueda solventar esta situación: no se puede aumentar la plantilla, es complejo limitar la demanda y no hay espacio físico para abrir nuevas salas de ecografía (en el supuesto de que se consiguieran equipos y radiólogos).

\hypertarget{alternativas}{%
\section{Alternativas}\label{alternativas}}

Dado que la situación no es sostenible, hay que buscar soluciones:

\begin{itemize}
\tightlist
\item
  \textbf{Controlar la demanda}: se va a trabajar en el tema, especialmente con los médicos de atención primaria. Es de esperar que esta medida tenga poco impacto y, el que tenga, tardará en verse.
\item
  \textbf{Dejar que otros hagan las ecografías}: esto es lo que está pasando. Con la excusa de la demora, algunos servicios están empezando a hacer estudios ecográficos. La inversión es escasa y con esto justifican la reestructuración de su actividad. Sin embargo, es probable queesta actividad no tenga impacto en las listas de espera, porque las peticiones de ecografías no las van a limitar. Y no solo esto, sino que está descrito que cuando médicos no radiólogos empiezan a hacer ecografías, la demanda aumenta. Como es de esperar, esto no lo cuentan así a las direcciones correspondientes. El tema se ``vende'' como que si ellos empiezan a hacer ecografías, va a mejorar la lista de espera. La realidad no creo que sea así, pero es muy difícil de argumentar y de discutir en Dirección.
\item
  \textbf{Optimizar la producción}: consiste en optimizar la capacidad de realización de estudios de las salas del área. Para ello, además de poder hacer actividad extra, se propone que los técnicos puedan hacer ecografías, controladas y supervisadas por los radiólogos.
\end{itemize}

\hypertarget{comentarios}{%
\section{Comentarios}\label{comentarios}}

La ecografía siempre se define como una técnica ``explorador dependiente'' y muy variable. Esto es así en un porcentaje de estudios ecográficos, pero no en todos. Los radiólogos estamos acostumbrados a adquirir la imagen y, al mismo tiempo, ir elaborando mentalmente el informe. Por ello, nos hemos acostumbrado a hacer unos estudios poco sistemáticos, enfocados a recoger las imágenes que consideramos más relevantes para nuestro diagnóstico. Esta forma de trabajar tiene muchas ventajas y ha permitido que la ecografía en manos de radiólogos expertos sea una técnica de gran rendimiento diagnóstico.

Sin embargo, no todas las ecografías son iguales, y no todas requieren la misma pericia a la hora de adquirir imágenes. Algunos estudios (como la ecografía cervical) son muy rutinarios y de fácil estandarización. Además, suelen ser poco gratificantes para el radiólogo, que ve más útil dedicar su tiempo a diagnosticar casos complejos con la ecografía en vez de medir dimensiones de una glándula tiroidea o adquirir imágenes de nódulos tiroideos. En estos casos, y solo en estos, y después de un periodo de formación riguroso, con definición de las competencias específicas, los técnicos pueden ser de gran ayuda.

\hypertarget{que-hacer-ahora}{%
\section{Qué hacer ahora}\label{que-hacer-ahora}}

La situación es la siguiente:

\begin{itemize}
\tightlist
\item
  Una demanda de ecografía que sobrepasa la capacidad de producción, sin posibilidad de aumentar la plantilla estructural y con poca capacidad de maniobra en lo que se refiere a incremento de jornadas de rebase.
\item
  Dificultad para controlar la demanda, aunque se trabajará en el tema.
\item
  Amenaza real por parte de otros servicios, que ya están haciendo ecografías.
\item
  Solicitud, por parte de la Dirección Gerencia del Distrito, de que se formen médicos de atención primaria para que realicen determinadas ecografías
\item
  Una plantilla de técnicos con deseo de realizar ecografías y que han demostrado ser competentes con una formación específica de corta duración.
\end{itemize}

En este escenario, hay que tomar algún tipo de decisión, ya que si no se realiza algún cambio o se toma alguna medida, es de esperar que alguien la tome por nosotros. Y cualquier acción que no venga desde Radiología no nos va a beneficiar.

\textbf{Opciones:}

\begin{itemize}
\tightlist
\item
  No hacer nada y usar la lista de espera como una medida de fuerza para que mejoren nuestras condiciones (más contratos). Es la reacción esperable, y que se puede plantear. Sin embargo, e independientemente de criterios éticos, no parece que sea razonable, ya que:

  \begin{itemize}
  \tightlist
  \item
    No es posible aumentar la plantilla
  \item
    No hemos podido, a pesar de que nos han dado medios, hacer más ecografías (estamos sobrepasados de actividad)
  \item
    No parece muy estimulante, en la situación actual, hacer más peonadas de ecografía.
  \item
    Por otra parte, siempre existe el peligro (ya visto en otras ocasiones y en otros centros) de que alguien vea esta situación como una oportunidad y se lance a hacer ecografías. La historia reciente de la radiología tiene ejemplos en los que los radiólogos fuimos perdiendo técnicas por nuestra actitud pasiva. La ecografía podría ser otro ejemplo.
  \end{itemize}
\item
  Aumentar la actividad que se externaliza. Se está haciendo ya, con cierta dificultad. No parece que vaya a tener un gran impacto ni a mejorar la situación global.
\item
  Rediseñar las agendas de actividad para aumentar la producción de ecografía. Esto supondría algún sacrificio de otras técnicas, algo que parece poco viable en la situación actual.
\item
  Contar con los técnicos para que las ecografías sigan haciéndose en radiología. Un posible escenario sería el siguiente:

  \begin{itemize}
  \tightlist
  \item
    Cuando exista un grupo de técnicos suficientes para realizar ecografías, se citan agendas específicas (tal como se explica más arriba) independientes de las agendas de radiólogos, pero (lógicamente) con un radiólogo asignado y que está realizando otra tarea.
  \item
    Los técnicos gestionan la agenda y hacen los estudios según los procedimientos establecidos. Consultan con los radiólogos cuando haya alguna duda. Las imágenes se guardan en el PACS y las listas se asignan a un radiólogo, que informa el estudio en otro momento. Esta actividad se recoge y retribuye como peonada.
  \end{itemize}
\end{itemize}

Las ventajas de esta última alternativa son muchas:

\begin{itemize}
\tightlist
\item
  Los estudios se hacen en Radiodiagnóstico. El radiólogo sigue siendo el propietario del proceso.
\item
  El gasto se reinvierte en Radiodiagnóstico. Genera puestos de trabajo para técnicos y mejora la nóminas de los radiólogos.
\item
  Abre las puertas a nuevas vías de desarrollo para los técnicos y para la instauración de nuevos cursos que pueden generar muchos beneficios para el servicio.
\item
  El proceso de aprendizaje estandarizado que se aplica para los técnicos se puede usar para los residentes. En los centros donde los técnicos hacen ecografía (muy pocos) la formación en ecografía de los técnicos no se ha visto mermada.
\item
  Nos permitiría crecer y facilitar nuevas inversiones.
\end{itemize}

En cualquier caso, es necesario tomar alguna medida que intente solucionar (al menos parcialmente) la situación de la ecografía. Se agradece cualquier tipo de sugerencia. En el contexto actual, mi apuesta es por el desarrollo del personal técnico y en la formación por competencias.

\bibliography{book.bib,packages.bib}


\end{document}
